%
% GNU courseware, XIN YUAN, 2017
%

\documentclass[11pt]{beamer}
\usepackage{beamerthemesplit}
\usepackage[BoldFont,SlantFont,CJKchecksingle]{xeCJK}
\setCJKmainfont[BoldFont=SimHei]{SimSun}
\setCJKmonofont{SimSun}
\usetheme{Madrid}
\usecolortheme{crane}

\parindent 2em

%title
\title{生物智能与算法}
\author{袁昕}
\date{\today}

\begin{document}

\frame{\titlepage}

\frame{
\centerline{\textbf{\Huge{总结和展望}}}
}

\frame{\frametitle{能解决的问题}
	\begin{itemize}
		\item<1-> 确定性的优化问题、预测问题,准确率有限,漏报率和准确率之间有关系。
		\item<2-> 多数只能找到次优解。对次优解的评价没有统一的标准。
		\item<3-> 可能比较适合研究问题本身在不断变化的问题,即环境不断变化,
如视觉认知,不同条件下成像的猫。
	\end{itemize}
}

\frame{\frametitle{研究思路}
	\begin{itemize}
		\item<1-> 各个已有生物智能方法之间的交叉结合。
		\item<2-> 与描述复杂、不确定现象的各种理论结合。
		\item<3-> 吸收生物学研究、脑科学研究的最新成果。
	\end{itemize}
}

\frame{\frametitle{理论研究}
	\begin{itemize}
		\item<1-> 研究复杂网络非线性动力学模型、随机过程。
		\item<2-> 研究多智能主体间的作用关系,包括兴奋和抑制、合作和竞争、正负反馈。
		\item<3-> 网络内部的节点和连接代表的事物可变,其作用规律也可变,研究异质节点连接的网络。
		\item<4-> 研究网络的结构不断变化的情形。
		\item<5-> 研究网络的网络,多网络间的相互作用。
	\end{itemize}
}

\frame{\frametitle{应用研究}
	\begin{itemize}
		\item<1-> 快速并行算法。
		\item<2-> 快速收敛算法。
		\item<3-> 逻辑推理的应用。
		\item<4-> 视觉认知的应用。
		\item<5-> 听觉认知的应用。
		\item<6-> 智慧城市的应用。
		\item<7-> 互联网和安全的应用。
	\end{itemize}
}

%end

\frame{\frametitle{结束}
\centerline{\textbf{\Huge{结束!}}}
}

\frame{\frametitle{版权申明}

本作品采用知识共享 署名-非商业性使用-禁止演绎 3.0 中国大陆 许可协议进行许可。
要查看该许可协议,可访问 http://creativecommons.org/licenses/by-nc-nd/3.0/cn/ 或者
写信到 Creative Commons, PO Box 1866, Mountain View, CA 94042, USA。
}

\end{document}
