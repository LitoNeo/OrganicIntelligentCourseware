%
% GNU courseware, XIN YUAN, 2017
%

\documentclass[11pt]{beamer}
\usepackage{beamerthemesplit}
\usepackage{algorithm, algorithmic}
\usepackage{tikz}
\usepackage[BoldFont,SlantFont,CJKchecksingle]{xeCJK}
\setCJKmainfont[BoldFont=SimHei,SlantedFont=KaiTi]{SimSun}
\setCJKsansfont[BoldFont=SimHei,SlantedFont=KaiTi]{SimSun}
\setCJKmonofont[ItalicFont={Adobe Fangsong Std}]{SimSun}
\setCJKfamilyfont{zhsong}{SimSun}
%\setCJKfamilyfont{zhhei}{Adobe Song Std}
\setCJKfamilyfont{zhhei}{SimHei}
%\setCJKfamilyfont{zhhei}{Adobe Heiti Std}
\setCJKfamilyfont{zhkai}{KaiTi}
\setCJKfamilyfont{zhfs}{FangSong}
\usetheme{berkeley}

\parindent 2em

%title
\title{生物智能与算法}
\author{袁昕}
\date{\today}

\begin{document}

\frame{\titlepage}

%tables

\section*{大纲}

\frame{\tableofcontents}

%summary

\section{思考}

\frame{\frametitle{思考}
	\begin{enumerate}
		\item<1-> 针对的问题是什么,除了优化问题和预测问题,还能处理其他什么问题?
		\item<2-> 如何对求解的量设计编码?
		\item<3-> 这一类方法的几何和物理本质是什么?
		\item<4-> 存在过度成熟问题吗?其优缺点和适用范围?
		\item<5-> 近年有哪些新的变种算法和模型?
		\item<6-> 有没有进一步改进的空间?
	\end{enumerate}
}

\section{基础}

\frame{
\centerline{\textbf{\Huge{基础}}}
}

\frame{\frametitle{科普}
	\begin{itemize}
		\item<1-> 图形摘要
		\item<2-> 自动控制的故事
		\item<3-> 电子搞基
		\item<4-> 折纸
	\end{itemize}
}

\frame{\frametitle{最优化}
	\begin{itemize}
		\item<1-> 问题描述(凸凹)
		\item<2-> 求解方法
		\item<3-> 正则化的本质(约束是什么)
		\item<4-> 泛函角度
		\item<5-> 物理角度
		\item<6-> 几何直观角度
	\end{itemize}
}

%topics

%
% GNU courseware, XIN YUAN, 2017
%

\section{进化算法}

\frame{
\centerline{\textbf{\Huge{进化算法}}}
}

\frame{\frametitle{定义}
又叫演化计算,是模拟自然界中的生物的演化过程产生的一种群体导向的随机搜索技术和方法。

~

是一种通用的问题求解方法,具有自组织、自适应、自学习性和本质并行性等特点,
不受搜索空间限制性条件的约束,也不需要其它辅助信息。
}

\frame{\frametitle{思想}
其算法是受生物进化过程中“优胜劣汰”的自然选择机制和遗传信息的传递规律的影响,
通过程序迭代模拟这一过程,把要解决的问题看作环境,在一些可能的解组成的种群中,
通过自然演化寻求最优解。
}

\frame{\frametitle{种类}
	\begin{itemize}
		\item<1-> 遗传算法
		\item<2-> 演化策略,修改参数
		\item<3-> 演化规划,自适应响应
		\item<4-> 遗传程序设计,改变分层节点和结构链接关系
		\item<5-> 多种群协同进化,竞争和合作的动力学系统
		\item<6-> 差分进化算法,个体间竞争和合作的系统
	\end{itemize}
}

%end


%
% GNU courseware, XIN YUAN, 2017
%

\section{免疫算法}

\frame{
\centerline{\textbf{\Huge{免疫算法}}}
}

\frame{\frametitle{定义}

\begin{block}{定义}
一种具有生成+检测 (generate and test) 的迭代过程的搜索算法。
\end{block}
}

\frame{\frametitle{思想}
生物免疫系统是一个分布式、自组织和具有动态平衡能力的自适应复杂系统。
它对外界入侵的抗原,可由分布全身的不同种类的淋巴细胞产生相应的抗体,
其目标是尽可能保证整个生物系统的基本生理功能得到正常运转。

~

具有较强模式分类能力,尤其对多模态问题的分析、处理和求解表现出较高的智能性和鲁棒性。
}

\frame{\frametitle{思想}
	\begin{itemize}
		\item<1-> 抗原和抗体
		\item<2-> 免疫疫苗
		\item<3-> 免疫算子
		\item<4-> 免疫调节
		\item<5-> 免疫记忆
		\item<6-> 抗原识别
	\end{itemize}
}

\frame{\frametitle{种类}
	\begin{itemize}
		\item<1-> 一般免疫算法
		\item<2-> 阴性选择和克隆选择算法
		\item<3-> 免疫网络学说与人工免疫网络模型
		\item<4-> 混合免疫算法
	\end{itemize}
}

\frame{\frametitle{发展}
在基于马尔科夫链的收敛性分析和非线性动力学模型等方面,
对免疫优化算法的非线性随机分析可能是未来研究的难点之一。

~

神经网络、内分泌及免疫这三大调节系统相互联系、相互补充和配合、相互制约的机理
为基于人工免疫系统的智能综合集成提供了生物学基础,
网络和智能成为免疫算法发展的不可缺少的特征,也是其重要应用领域。

~

免疫算法能增强系统的鲁棒性,在网络、智能系统和鲁棒系统中的应用。
}

%end


%
% GNU courseware, XIN YUAN, 2017
%

\section{群聚智能}

\frame{
\centerline{\textbf{\Huge{群聚智能}}}
}

\frame{\frametitle{定义}

无智能或简单智能的主体通过任何形式的聚集协作而表现出智能行为的特性。

~

群,是一组相互之间可以进行直接或间接通信(通过改变局部环境)的主体。
}

\frame{\frametitle{思想}

源于分子自动机系统的自组织研究。
在没有集中且不提供全局模型的前提下,为寻找复杂分布式问题解决方案提供基础。
}

\frame{\frametitle{思想}
相对于传统的梯度优化算法

~

	\begin{itemize}
		\item<1-> 完全分布式处理个体间和个体环境间作用,具有自组织性;
		\item<2-> 个体间非直接,合作基于环境感知,系统可扩展和安全;
		\item<3-> 无集中控制,具有鲁棒性,个别个体失效不会影响整个问题求解;
		\item<4-> 个体智能简单,实现方便,执行时间短。
	\end{itemize}
}

\frame{\frametitle{种类}
	\begin{itemize}
		\item<1-> 蚁群(蚁狮)
		\item<2-> 鸟群(粒子群)
		\item<3-> 人工鱼群
		\item<4-> 蜂群
		\item<5-> 蛙群
		\item<6-> 萤火虫
	\end{itemize}
}

\frame{\frametitle{种类}
	\begin{itemize}
		\item<1-> 蝙蝠群
		\item<2-> 花朵授粉
		\item<3-> 灰狼
		\item<4-> 猫群?
		\item<5-> 鲸鱼捕食
	\end{itemize}
}

\frame{\frametitle{公式示例}

\begin{equation*}
p_{ij}^{k}(t) = \left\{
	\begin{array}{lc}
	\frac{[\tau_{ij}(t)]^{\alpha}\cdot[\eta_{ij}(t)]^{\beta}}{\sum\limits_{s\in J_{k}(i)} [\tau_{is}(t)]^{\alpha}\cdot[\eta_{is}(t)]^{\beta}}, & \mbox{如果} J \in J_{k}(i) \\
	0, & \mbox{否则}
	\end{array} \right.
\end{equation*}
}

\frame{\frametitle{定理示例}

\newtheorem{zh-thm}{定理}
\begin{zh-thm}
如果有a,b,c, 则有$a^2+b^2=c^2$。
\end{zh-thm}

\renewcommand\proofname{证明}
\begin{proof}
因为有a,b,c, 所以有$a^2+b^2=c^2$。
\end{proof}
}

\frame{\frametitle{算法示例}

\begin{algorithm}[H]
\caption{计算 $y=x^n$}\label{alg1}
\algsetup{linenosize=\tiny} \scriptsize
	\begin{algorithmic}
		\REQUIRE $n \geq 0 \vee x \neq 0$
		\ENSURE $y=x^n$
		\STATE $y \Leftarrow 1$
		\IF{$n < 0$}
			\STATE $X \Leftarrow 1 / x$
			\STATE $N \Leftarrow -n$
		\ELSE
			\STATE $X \Leftarrow x$
			\STATE $N \Leftarrow n$
		\ENDIF
		\WHILE{$N \neq 0$}
			\IF{$N$ is even}
				\STATE $X \Leftarrow X \times X$
				\STATE $N \Leftarrow N / 2$
			\ELSE[$N$ is odd]
				\STATE $y \Leftarrow y \times X$
				\STATE $N \Leftarrow N - 1$
			\ENDIF
		\ENDWHILE
	\end{algorithmic}
\end{algorithm}
}

\frame{\frametitle{流程图示例}

\tikzstyle{block}=[rectangle,draw,fill=blue!20,text width=4em,text centered,rounded corners]
\tikzstyle{huge-block}=[rectangle,draw,fill=cyan!20,text width=5em,text centered,rounded corners,minimum height=4em]
\tikzstyle{line}=[draw]
\begin{tikzpicture}[node distance=2cm, auto]
\path[use as bounding box] (-4, 0) rectangle (10, -2);
\path[line]<1-> node[huge-block](center){群聚智能};
\path[line, <->]<2-> node[block, below of=center, node distance=3cm](center-develop){蚁群算法}
(center)--(center-develop);
\path[line, <->]<3-> node[block, left of=center-develop, node distance=3cm](left-develop){粒子群}
(center)--(left-develop);
\path[line, <->]<4-> node[block, right of=center-develop, node distance=3cm](right-develop){灰狼}
(center)--(right-develop);
\end{tikzpicture}
}

\frame{\frametitle{流程图示例}

\begin{tikzpicture}[font={\sf \scriptsize}]

\node (start) at (-2,3) [draw, terminal, minimum width=2cm, minimum height=0.5cm] {开始};
\node (getdata) at (-2,2) [draw, predproc, align=left, minimum width=2cm, minimum height=0.5cm] {读取数据};
\node (init) at (-2,1) [draw, process, minimum width=3cm, minimum height=0.5cm] {设置参数和初始化};
\coordinate (point1) at (-0.25,1);
\node (eval) at (1,1) [draw, process, minimum width=2cm, minimum height=0.5cm] {评价蚁群};
\node (decide) at (1,-1) [draw, decision, minimum width=2cm, minimum height=0.5cm] {判断终止条件};
\node (storage) at (1,-3) [draw, storage, minimum width=3cm, minimum height=0.5cm] {输出结果并存储};
\node (iter) at (4,0) [draw, process, minimum width=3cm, minimum height=0.5cm] {迭代次数增加};
\node (select) at (4,0.7) [draw, process, minimum width=3cm, minimum height=0.5cm] {概率选择移动方向};
\node (update) at (4,1.4) [draw, process, minimum width=3cm, minimum height=0.5cm] {信息素更新};
\coordinate (point2) at (4,2);
\node (end) at (4,-3) [draw, terminal, minimum width=2cm, minimum height=0.5cm] {结束};

\draw[->] (start) -- (getdata);
\draw[->] (getdata) -- (init);
\draw[->] (init) -- (point1) -- (eval);
\draw[->] (eval) -- (decide);
\draw[->] (decide) -| node[right]{否} (iter);
\draw[->] (decide) -- node[right]{是} (storage);
\draw[->] (iter) -- (select);
\draw[->] (select) -- (update);
\draw[->] (update) -- (point2) -| (point1);
\draw[->] (storage) -- (end);
\end{tikzpicture}
}

%end


%
% GNU courseware, XIN YUAN, 2017
%

\section{觅食算法}

\frame{
\centerline{\textbf{\Huge{觅食算法}}}
}

\frame{\frametitle{思想和种类}

{\CJKfamily{zhfs}
来自动物\textbf{高效觅食}的启发:
}

~

\begin{itemize}
	\item<2-> 布朗运动
	\item<3-> 随机游走
	\item<4-> 随机微分方程和伊藤过程
	\item<5-> 轻尾和重尾分布
	\item<6-> 莱维过程(L\'evy Process)
	\item<7-> 分解定理(布朗+泊松+平方可积离散鞅,可类比PID)
\end{itemize}
}

%end


%
% GNU courseware, XIN YUAN, 2017
%

\section{人工神经网络}

\frame{
\centerline{\textbf{\Huge{人工神经网络}}}
}

\frame{\frametitle{定义}

人工神经网络是由大量处理单元互联组成的非线性、自适应信息处理系统。
}

\frame{\frametitle{思想}

在现代神经科学研究成果的基础上提出,试图通过模拟大脑神经网络处理、记忆信息的方式进行信息处理。

~

非线性、非局限性、非定常性、非凸性。

~

自适应、自组织、自学习。
}

\frame{\frametitle{思想}

基于逻辑符号和规则推理的专家系统,在处理直觉、非结构化信息方面有缺陷。

~

人工神经网络是连接主义的观点,是神经元相互联接而成的自适应非线性动态系统,
分为前向网络(有向无环图)和反馈网络(无向完备图)两类。

~

理论研究:MP模型、Hebb规则、感知器、适应谐振ART理论、自组织映射、认知机网络、非线性动力学。
}

\frame{\frametitle{种类}
	\begin{itemize}
		\item<1-> Hopfield
		\item<2-> 波尔兹曼机(模拟退火)
		\item<3-> 联想记忆
		\item<4-> BP
		\item<5-> 局部逼近神经网络(CMAC,RBF,B样条)
		\item<6-> 自组织SOM
		\item<7-> ART
		\item<8-> SVM
		\item<9-> 强化学习
	\end{itemize}
}

\frame{\frametitle{种类}
	\begin{itemize}
		\item<1-> 集成式网络(boosting,梯度推进机,决策树,随机森林)
		\item<2-> 量子神经网络
		\item<3-> 脉冲耦合
		\item<4-> 混沌神经网络
		\item<5-> 深度学习(卷积、循环、递归,置信网络,生成式对抗网络,迁移学习,果蝇的大脑)
	\end{itemize}
}

%end


%
% GNU courseware, XIN YUAN, 2017
%

\section{复杂网络}

\frame{
\centerline{\textbf{\Huge{复杂网络}}}
}

\frame{\frametitle{定义}

具有自组织、自相似、吸引子、小世界、无标度中部分或全部性质的网络称为复杂网络。
}

\frame{\frametitle{思想}

互联网拓扑、社交网络、脑神经网络、基因调控网络、蛋白质调控网络、生态系统、地球演化、N体作用。
}

\frame{\frametitle{思想}
	\begin{itemize}
		\item<1-> 结构复杂:节点数巨大,结构具有多种不同特征。
		\item<2-> 网络进化:表现在节点或连接的产生与消失,网络结构不断变化。
		\item<3-> 连接多样性:节点之间的连接权重存在差异,且有可能存在方向性。
		\item<4-> 动力学复杂性:节点集可能属于非线性动力学系统,节点状态随时间发生复杂变化。
		\item<5-> 节点多样性:节点和连接可代表任何事物。
		\item<6-> 多重复杂性融合:多重复杂性相互影响,结果难以预测。
	\end{itemize}
}

\frame{\frametitle{思想}
网络的几何性质,网络的形成机制,网络演化的统计规律,网络上的模型性质,
网络的结构稳定性,网络的演化动力学机制。
}

\frame{\frametitle{种类}
	\begin{itemize}
		\item<1-> 马尔科夫场
		\item<2-> 贝叶斯网络
		\item<3-> 社交网络
		\item<4-> 视觉听觉网络
		\item<5-> 大脑生理病理网络
		\item<6-> 组分调控网络
		\item<7-> 生态系统网络(含传染病网络)
		\item<8-> 网络安全攻防
		\item<9-> 天文N体系统
	\end{itemize}
}

%end


%end

\frame{\frametitle{结束}
\centerline{\textbf{\Huge{结束!}}}
}

\frame{\frametitle{版权申明}

本作品采用知识共享 署名-非商业性使用-禁止演绎 3.0 中国大陆 许可协议进行许可。
要查看该许可协议,可访问 http://creativecommons.org/licenses/by-nc-nd/3.0/cn/ 或者
写信到 Creative Commons, PO Box 1866, Mountain View, CA 94042, USA。
}

\end{document}
