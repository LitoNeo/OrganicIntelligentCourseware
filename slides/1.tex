%
% GNU courseware, XIN YUAN, 2018
%

\documentclass[11pt]{beamer}
\usepackage{beamerthemesplit}
\usepackage[BoldFont,SlantFont,CJKchecksingle]{xeCJK}
\setCJKmainfont[BoldFont=SimHei,SlantedFont=KaiTi]{SimSun}
\setCJKsansfont[BoldFont=SimHei,SlantedFont=KaiTi]{SimSun}
\setCJKmonofont[ItalicFont={Adobe Fangsong Std}]{SimSun}
\setCJKfamilyfont{zhsong}{SimSun}
%\setCJKfamilyfont{zhhei}{Adobe Song Std}
\setCJKfamilyfont{zhhei}{SimHei}
%\setCJKfamilyfont{zhhei}{Adobe Heiti Std}
\setCJKfamilyfont{zhkai}{KaiTi}
\setCJKfamilyfont{zhfs}{FangSong}
\usetheme{goettingen}
\usecolortheme{beetle}

\parindent 2em

%title
\title{生物智能与算法}
\author{袁昕}
\date{\today}

\begin{document}

\frame{\titlepage}

%tables

\section*{大纲}

\frame{\tableofcontents}

%summary

\section{基础}

\frame{
\centerline{\textbf{\Huge{基础}}}
}

\frame{\frametitle{科普}
	\begin{itemize}
		\item<1-> 图形摘要
		\item<2-> 自动控制的故事
		\item<3-> 电子搞基
		\item<4-> 折纸
	\end{itemize}
}

\frame{\frametitle{最优化}
	\begin{itemize}
		\item<1-> 问题描述
		\item<2-> 求解方法
		\item<3-> 正则化的本质(约束是什么)
		\item<4-> 泛函角度
		\item<5-> 物理角度
		\item<6-> 几何直观角度(微分几何)
	\end{itemize}
}

\frame{\frametitle{问题描述}
	\begin{itemize}
		\item<1->凸函数和非凸函数
		\item<2->线性函数和非线性函数
		\item<3->有限维空间和无限维空间
		\item<4->离散函数(如组合优化)和连续函数(能量泛函及最大似然)
	\end{itemize}
}

%end

\frame{\frametitle{结束}
\centerline{\textbf{\Huge{结束!}}}
}

\frame{\frametitle{版权申明}

本作品采用知识共享 署名-非商业性使用-禁止演绎 3.0 中国大陆 许可协议进行许可。
要查看该许可协议,可访问 http://creativecommons.org/licenses/by-nc-nd/3.0/cn/ 或者
写信到 Creative Commons, PO Box 1866, Mountain View, CA 94042, USA。
}

\end{document}
